\section{Méthodes spéciales}
\index{méthode spéciale}
Python est un langage avec une syntaxe de haut niveau, c'est-à-dire facilement compréhensible par l'utilisateur humain. Derrière cette syntaxe se cachent des méthodes appelées méthodes spéciales. On les reconnaît par la présence de soulignés qui encadrent leur nom, c'est le cas de \mintinline{python3}{__new__()} et \mintinline{python3}{__init__()}.

\paragraph{Exemples}
\begin{minted}{python3}
## Syntaxe de haut niveau   ## Méthode spéciale correspondante
# Listes
L = [1, 2, 3]
2 in L                      # L.__contains__(2)
len(L)                      # L.__len__()

# Opérations
objet1 + objet2             # objet1.__add__(objet2)
objet1 == objet2            # objet1.__eq__(objet2)

# Conversion en str ou affichage
print(objet)                # objet.__str__()
str(objet)                  # objet.__str__()

# Appel d'une fonction
fonction()                  # fonction.__call__() <- intéressant celui-là !
\end{minted}

\paragraph{Plus d'informations} \href{https://docs.python.org/3/reference/datamodel.html#specialnames}{Documentation Python 3}, \href{https://openclassrooms.com/courses/apprenez-a-programmer-en-python/les-methodes-speciales-1}{OpenClassroom}


\subsection{Création, initialisation et finalisation}
\begin{description}
    \item[\mintinline{python3}{Objet.__new__(cls[, *args, **kwargs])}]~

    Créateur de l'instance. C'est une méthode statique qui prend en premier paramètre obligatoire la classe de l'instance à créer. Les arguments suivants sont ceux passés à l'initialiseur. Cette méthode doit renvoyer une nouvelle instance. L'initialiseur est alors appelé. En général, on n'implémente pas cette méthode, sauf dans certains cas particuliers.\bigskip

    Cette méthode statique est un cas particulier qui ne nécessite pas de décorateur \mintinline{python3}{@staticmethod}.

    \item[\mintinline{python3}{Objet.__init__(self[, *args, **kwargs])}]~

    Initialiseur de l'instance. C'est ici qu'on initialise les attributs de l'instance.
\end{description}

Ces deux méthodes forment le constructeur\label{constructeur} de l'instance, elles sont appelées lorsqu'on appelle la classe pour construire un objet. Si on surcharge ces méthodes, il ne faut pas oublier d'appeler les méthodes héritées grâce à \mintinline{python3}{super()}. Dans le cas où la classe hérite uniquement d'\mintinline{python3}{object}, il n'y a pas besoin d'appeler \mintinline{python3}{super().__init__()} car les instances d'\mintinline{python3}{object} n'ont aucun attribut (donc il n'y a pas d'initialisation).

\paragraph{Exemple} On veut définir une classe \og singleton \fg{} qui ne peut créer qu'une instance.

\begin{minted}{python3}
class Singleton:
    """Classe qui ne peut instancier qu'une fois."""

    instance = None

    def __new__(cls, *args, **kwargs):
        if not cls.instance:
            cls.instance = super().__new__(cls)
            return cls.instance
        else:
            raise TypeError(f"Cette classe singleton possède déjà une instance : {cls.instance}")

    def __del__(self):
        Singleton.instance = None

    def __init__(self, *args, **kwargs):
        pass
\end{minted}

\begin{description}
    \item[\mintinline{python3}{Objet.__del__(self)}]~

    Finaliseur de l'instance. Cette méthode est appelée lorsqu'un objet est sur le point d'être détruit, mais n'est pas responsable de sa destruction. Lorsqu'on hérite uniquement d'\mintinline{python3}{object}, il n'est pas nécessaire d'appeler \mintinline{python3}{super().__del__()} car elle ne fait rien. La syntaxe \mint{python}|del obj| décrémente le nombre de références de \mintinline{python3}{obj}. Si celui-ci atteint zéro, alors \mintinline{python3}{obj.__del__()} est appelée.
\end{description}

\subsection{Représentation et chaîne de caractères d'un objet}
Si on crée une instance de la classe \mintinline{python3}{Singleton} précédente et qu'on demande sa représentation dans l'interpréteur, il nous renvoie quelque chose de pas très explicite:

\begin{minted}{pycon}
>>> inst = Singleton()
>>> inst
<__main__.Singleton object at 0x000002B07DB73898>
\end{minted}

On doit donc définir une méthode de représentation.

\begin{description}
    \item[\mintinline{python3}{Objet.__repr__(self)}]~

    Appelée par \mintinline{python3}{repr()}, ou bien lorsqu'on demande l'objet dans l'interpréteur. Cette méthode calcule et renvoie une chaîne de caractères compréhensible \og officielle \fg{}, c'est-à-dire qu'elle doit ressembler à une expression Python à partir de laquelle on doit pouvoir recréer un objet de la même valeur. Si ce n'est pas possible, elle devrait renvoyer une description entre chevrons \mintinline{python3}{"<description>"}. 
\end{description}

Parfois, on veut quelque chose de plus joli destiné à un véritable affichage (quand on appelle \mintinline{python3}{print()}). On peut vouloir aussi avoir la capacité de convertir un objet en une chaîne de caractère avec \mintinline{python3}{str()}. On doit alors définir une autre méthode spéciale:

\begin{description}
    \item[\mintinline{python3}{Objet.__str__(self)}]~

    Appelée par \mintinline{python3}{str()}, \mintinline{python3}{print()} et \mintinline{python3}{format()}. Cette méthode renvoie une chaîne de caractère correspondant à la représentation informelle de l'objet. Si cette méthode n'est pas définie, alors \mintinline{python3}{__repr__()} est utilisée à la place.
\end{description}
\paragraph{Exemple} L'exemple suivant
\begin{minted}{python3}
class MaClasse:
    def __init__(self, attr):
         self.attribut = attr

    def __repr__(self):
        return "MaClasse(attribut={})".format(self.attribut)

    def __str__(self):
        return "Instance de MaClasse ayant comme attribut {}"
              .format(self.attribut)
\end{minted}
permet de faire:
\begin{minted}{pycon}
>>> obj = MaClasse("Exemple")
>>> obj
MaClasse(attribut=Exemple)
>>> print(obj)
Instance de MaClasse ayant comme attribut Exemple.
\end{minted}

\subsection{Accès et modification des attributs}
On a vu précédemment les propriétés qui permettent une sorte d'encapsulation des attributs. Lorsqu'on veut accéder à un attribut par la syntaxe \mint{python}|instance.attr| Python appelle en premier une méthode spéciale.

\begin{description}
    \item[\mintinline{python3}{Objet.__getattribute__(self, name)}]~
    
    Appelée en premier lorsque l'on veut accéder à un attribut par les syntaxes
    \begin{minted}{python3}
instance.attr             # instance.__getattribute__('attr')
getattr(instance, 'attr') # instance.__getattribute__('attr')
    \end{minted} 
    Cette méthode doit renvoyer l'attribut \mintinline{python3}{name} demandé s'il existe (ou calculer sa valeur) ou lever une exception \mintinline{python3}{AttributeError} sinon. Dans ce cas, la méthode \mintinline{python3}{__getattr__()} est appelée.\medskip
    
    Cette méthode est définie dans la classe \mintinline{python3}{object}, le mécanisme d'accès par défaut aux attributs est
    donc le suivant:
    \begin{enumerate}
        \item \mintinline{python3}{object.__getattribute__()} commence par rechercher \mintinline{text}{name} sous forme de \hyperref[sec:descripteur]{descipteur} dans le dic\-tion\-naire \mintinline{python3}{__dict__} de la classe de l'instance (et de ses classes parentes s'il ne trouve pas). Elle essaie en fait d'appeler la méthode \mintinline{python3}{__get__()} de \mintinline{text}{name}.
        \item \mintinline{python3}{object.__getattribute__()} recherche ensuite \mintinline{text}{name} sous forme de simple variable dans le dictionnaire \mintinline{python3}{__dict__} de l'instance, puis dans celui de sa classe si elle ne le trouve pas, ainsi que dans celui de chaque classe parente jusqu'à le trouver.
        \item Si \mintinline{python3}{object.__getattribute__()} n'a pas trouvé \mintinline{text}{name} dans aucun \mintinline
        {python3}{__dict__}, elle lève une exception\break \mintinline{python3}{AttributeError}.
    \end{enumerate}

    Surcharger cette méthode va donc modifier le mécanisme par défaut. Cependant, pour les attributs dont on ne veut pas
    modifier l'accès, il faut penser à appeler la méthode \mintinline{python3}{__getattribute__()} d'\mintinline{python3}{object}
    ou de la classe parente.

    \item[\mintinline{python3}{Objet.__getattr__(self, name)}]~

    Appelée si \mintinline{python3}{__getattribute__()} lève une exception \mintinline{python3}{AttributeError}. Par défaut, cette méthode fait la même chose, mais c'est ici que l'on peut calculer des attributs dynamiques: des attributs qui ne sont pas initialisés mais dont on veut pouvoir calculer la valeur.

    \item[\mintinline{python3}{Objet.__setattr__(self, name, value)}]~

Appelée lorsque l'on assigne une valeur à un attribut: 
    \begin{minted}{python3}
instance.attr = value            # instance.__setattr__('attr', value)
setattr(instance, 'attr', value) # instance.__setattr__('attr', value)
    \end{minted}
    
    Cette méthode est définie dans la classe \mintinline{python3}{object}, l'ordre de recherche de l'attribut à modifier
    est le suivant:
    \begin{enumerate}
        \item Si un descripteur est trouvé, il est utilisé pour modifier la valeur de \mintinline{text}{name}.
        \item Sinon une nouvelle clé est créée dans le \mintinline{python3}{__dict__} de l'instance avec pour valeur
        \mintinline{text}{value}.
    \end{enumerate}

    Surcharger cette méthode va donc modifier le mécanisme par défaut. Cependant, pour les attributs dont on ne veut pas
    modifier le comportement de modification, il faut penser à appeler la méthode \mintinline{python3}{__setattr__()} d'\mintinline{python3}{object} ou de la classe parente.

    \item[\mintinline{python3}{Objet.__delattr__(self, name)}]~

    Appelée lorsque l'on veut détruire un attribut : 
    \begin{minted}{python3}
del instance.attr         # instance.__delattr__('attr')
delattr(instance, 'attr') # instance.__delattr__('attr')
    \end{minted}
    
    Finalise l'attribut avant sa suppression s'il existe et lève une exception \mintinline{python3}{AttributeError} sinon. Cette méthode doit appeler \mintinline{python3}{super().__delattr__()} pour éviter une récursivité infinie lors de l'appel à la suppression de l'attribut.
\end{description}  


\paragraph{Exemple}
\begin{minted}{python3}
class MaClasse:
    def __init__(self)
        self.a = int()

    def __getattribute__(self, attribut):
        print("J'accède à l'attribut {}...".format(attribut))
        return object.__getattribute__(self, attribut)

    def __getattr__(self, attribut):
        print("L'attribut {} est inaccessible !".format(attribut))

    def __setattr__(self, attribut, valeur):
        object.__setattr__(self, attribut, valeur)
        print("L'attribut a été changé !")
        # Il est nécessaire d'appeler la méthode par défaut, car appeler self.__setattr__
        # donnerait une récursivité infinie. En fait, on ne sait à ce stade pas comment
        # Python change concrètement la valeur de l'attribut.
\end{minted}

Cela permet de faire:
\begin{minted}{pycon}
>>> objet = MaClasse()
L'attribut a été changé !
>>> objet.b
L'attribut b est inaccessible !
>>> objet.a
J'accède à l'attribut a...
0
>>> objet.attribut = 1
L'attribut a été changé !
>>> objet.a
J'accède à l'attribut a...
1
\end{minted}

\subsection{Surcharges d'opérateur}
\index{surcharge d'opérateur}
Les surcharges d'opérateur permettent de faire des opérations arithmétiques avec des objets, c'est-à-dire d'indiquer à Python ce qu'il faut faire lorsque l'on exécute \mintinline{python3}{objet1 + objet2}. Ces méthodes prennent en arguments \mintinline{python3}{self} (l'objet 1) et l'objet 2.
\begin{center}
\begin{tabular}{|p{2.5cm}|c|}
        \hline
        \multicolumn{1}{|c|} {\bf Méthode} & {\bf Appel}\\
        \hline
        \mintinline{python3}{__add__()}\index{\_\_add\_\_()} & \mintinline{python3}{objet1 + objet2}\\
        \hline
        \mintinline{python3}{__sub__()}\index{\_\_sub\_\_()} & \mintinline{python3}{objet1 - objet2}\\
        \hline
        \mintinline{python3}{__mul__()}\index{\_\_mul\_\_()} & \mintinline{python3}{objet1 * objet2}\\
        \hline
        \mintinline{python3}{__truediv__()}\index{\_\_truediv\_\_()} & \mintinline{python3}{objet1 / objet2}\\
        \hline
        \mintinline{python3}{__floordiv__()}\index{\_\_floordiv\_\_()} & \mintinline{python3}{objet1 // objet2}\\
        \hline
        \mintinline{python3}{__mod__()}\index{\_\_mod\_\_()} & \mintinline{python3}{objet1 \% objet2}\\
        \hline
\end{tabular}
\end{center}
Les deux objets ne sont pas nécessairement du même type ! Cependant, cette opération n'est pas symétrique : le code \mintinline{python3}{objet + 5} par exemple exécute \mintinline{python3}{objet.__add__(5)}, alors que \mintinline{python3}{5 + objet} exécute \mintinline{python3}{int.__add__(5)}. Pour que l'opération soit symétrique, il faut aussi définir ces fonctions avec le préfixe \mintinline{python3}{r} (par exemple \mintinline{python3}{__radd__()}).