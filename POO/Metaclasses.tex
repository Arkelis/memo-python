\section{Métaclasses}
\subsection{Principe}
\index{métaclasse}\label{sec:metaclasses}
Les métaclasses sont les classes qui instancient d'autres classes. Par défaut, une seule métaclasse est définie : la métaclasse \mintinline{python3}{type}. On s'en rend compte en demandant le type des classes que l'on crée.

\begin{minted}{pycon}
>>> class MaClasse:
...     pass
...
>>> type(MaClasse)
<class 'type'>
\end{minted}

\subsection[La métaclasse type]{La métaclasse \mintinline[fontsize=\normalsize]{python3}{type}}

On sait que passer à \mintinline{python3}{type()} un objet renvoie son type, c'est-à-dire l'objet qui l'a instancié. Mais
\mintinline{python3}{type()} permet aussi des créer des types (des classes) à la volée si on lui passe plus d'arguments :
un nom, un itérable contenant ses classes parentes et un dictionnaire contenant ses attributs.

\begin{minted}{pycon}
>>> MyType = type("MyType", (), {}) # une classe on ne peut plus basique
>>> MyType()
<__main__.MyType object at 0x7ff838fb7518>
>>> issubclass(MyType, object)
True
>>> isisntance(MyType, type)
True
\end{minted}

On peut donc créer des fonctions qui créent des classes.

\begin{minted}{python3}
def class_creator(name, bases=(), attrs={}): # il faut garder la signature de type()
    """Créateur de classe personnalisé.

    Celui ajoute à chaque classe créée un identifiant de classe correspondant au
    nombre de classes créées au moment de l'appel de class_creator().
    """
    if not hasattr(class_creator, "increment"):
        class_creator.increment = 0 # on utilise un attribut de fonction
    attrs["class_id"] = class_creator.increment
    class_creator.increment += 1
    return type(name, bases, attrs)
\end{minted}

\begin{minted}{pycon}
>>> first = class_creator("first")
>>> second = class_creator("second")
>>> first.class_id
0
>>> second.class_id
1
>>> class Third(metaclass=class_creator):
...     pass
...
>>> Third.class_id
2
\end{minted}

Ou plutôt des générateurs de classes :

\begin{minted}{python3}
def class_generator_function(bases=(), attrs={}):
    """Générateur de classe."""
    increment = 0
    name = yield # initialisation
    while True:
        attrs["class_id"] = increment
        name = yield type(name, bases, attrs)
        increment += 1
\end{minted}

\begin{minted}{pycon}
>>> class_generator = class_generator_function()
>>> next(class_generator) # initialisation
>>> first = class_generator.send("first")
>>> second = class_generator.send("second")
>>> first.class_id
0
>>> second.class_id
1
\end{minted}

\subsection{Ecrire une métaclasse}

On est un peu limité dans le cas de fonctions qui créent des classes. Pour des choses plus complexes, on peut écrire
des classes qui instancient d'autres classes : des métaclasses.\medskip

Pour qu'une classe puisse instancier d'autres classes, il faut hériter de \mintinline{python3}{type}. Cela permet notamment d'hériter de sa fonction \mintinline{python3}{__new__()}. D'habitude (c'est-à-dire quand on hérite simplement de \mintinline{python3}{object}), cette fonction ne fait rien de spécial, elle retourne simplement un objet vide que l'on initialise dans \mintinline{python3}{__init__()}. Dans le cas de \mintinline{python3}{type}, c'est cette fonction qui est chargée de créer la classe. Pour indiquer que l'on est en train de définir une métaclasse, on écrit \mintinline{python3}{cls} au lieu de \mintinline{python3}{self} pour faire référence à l'objet instancié, et \mintinline{python3}{mcls} au lieu de \mintinline{python3}{cls} pour faire référence au type.\medskip

Cette métaclasse ne fait rien de plus que \mintinline{python3}{type} :

\begin{minted}{python3}
class SimpleMeta(type):
    def __new__(mcls, name, bases, attrs):
        print(name, "was created.")
        return super().__new__(mcls, name, bases, attrs)

class ClassUsingSimpleMeta(metaclass=SimpleMeta):
    pass
\end{minted}

A l'exécution on verra : \mint{text}|ClassUsingSimpleMeta was created.|

Si l'on veut le même comportement que les exemples précédents :

\begin{minted}{python3}
class SimpleMeta(type):
    class_id = 0
    def __init__(cls, name, bases, attrs): # on récupère les arguments de __new__
        super().__init__(name, bases, attrs)
        cls.class_id = cls.class_id
        type(cls).class_id += 1
\end{minted}

\begin{minted}{pycon}
>>> class Class0(metaclass=SimpleMeta):
...     pass
...
>>> Class2 = SimpleMeta("Class2", (), {})
>>> Class0.class_id
0
>>> Class1.class_id
1
\end{minted}

On peut ajouter des paramètres lors de la déclaration d'une classe :

\begin{minted}{python3}
class AbstractoOrNotAbstractMeta(type):
    def __new__(mcls, name, bases, attrs, abstract=False):
        cls = super().__new__(mcls, name, bases, attrs)
        print(name)
        if abstract:
            def new(cls, *args, **kwargs):
                raise TypeError("This class is abstract.")
            cls.__new__ = new
        else:
            cls.__new__ = object.__new__
        return cls


class Abstract(metaclass=AbstractoOrNotAbstractMeta, abstract=True):
    pass

class Concrete(Abstract):
    pass

Concrete()
Abstract() # TypeError
\end{minted}

\subsection{Application des métaclasses : propriété de classe}

On pourrait imaginer des propriétés de classes afin d'ajouter une couche de logique sur une simple variable de classe. Au lieu de définir un descripteur générique, on créer une métaclasse qui aura comme propriété la future propriété de classe.

\paragraph{Exemple} Un exemple simple
\begin{minted}{python3}
class ClassPropertyMeta(type):
    def __new__(mcs, name, bases, attrs):
        """Créateur personnalisé.
        
        On redéfinit __new__ pour s'assurer que les éventuels setters des
        propriétés soient appelés.
        """
        cls = super().__new__(mcs, name, bases, {})
        for attr, value in attrs.items():
            setattr(cls, attr, val)
        return cls

    @property
    def some_positive_attr(self):
        return self._propriete

    @some_positive_attr.setter
    def some_positive_attr(self, value):
        if value < 0:
            raise ValueError("some_positive_integer must be > 0.")
        self._propriete = value


class ClassPropertyOwner(metaclass=ClassPropertyMeta):
    some_positive_attr = -1

\end{minted}

Cette définition de classe va lever une exception \mintinline{python3}{ValueError}. Oui, une déclaration
de classe peut lever une exception.
