\section{Métaclasses}
\subsection{Principe}
\index{métaclasse}\label{sec:metaclasses}
Les métaclasses sont les classes qui instancient d'autres classes. Par défaut, une seule métaclasse est définie : la métaclasse \mintinline{python3}{type}. On s'en rend compte en demandant le type des classes que l'on crée.

\begin{minted}{pycon}
>>> class MaClasse:
...     pass
...
>>> type(MaClasse)
<class 'type'>
\end{minted}

\subsection[La métaclasse type]{La métaclasse \mintinline[fontsize=\normalsize]{python3}{type}}

On sait que passer à \mintinline{python3}{type()} un objet renvoie son type, c'est-à-dire l'objet qui l'a instancié. Mais
\mintinline{python3}{type()} permet aussi des créer des types (des classes) à la volée si on lui passe plus d'arguments.

\subsection{Application: propriété de classe}
On pourrait imaginer des propriétés de classes afin d'ajouter une couche de logique sur une simple variable de classe. Au lieu de définir un descripteur générique, on créer une métaclasse qui aura comme propriété la future propriété de classe.

\paragraph{Exemple} Un exemple simple
\begin{minted}{python3}
class MaMetaclasse(type):
    @property
    def propriete(self):
        return self._propriete
    @propriete.setter
    def propriete(self, value):
        self._propriete = value

class MaClasse(metaclass=MaMetaclasse):
    propriete = 5

print(MaClasse.propriete) # 5
\end{minted}
