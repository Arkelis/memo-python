\section{Classes}
\index{classe}

Les classes permettent des créer des objets\index{objet} appelés instances\index{instance} qui partagent des caractéristiques communes. Une classe est en fait un gabarit qui nous permet de créer un certain type d'objets.

\paragraph{Plus d'informations} \href{https://openclassrooms.com/courses/apprenez-a-programmer-en-python/premiere-approche-des-classes}{Introduction OpenClassrooms}, \href{https://docs.python.org/fr/3/tutorial/classes.html}{Documentation Python 3}, \href{https://fr.wikibooks.org/wiki/Programmation_Python/Classes#Définition_d'une_classe_élé mentaire}{Wikilivres}

\subsection{Structure d'une classe}
Les objets d'une classe partagent des caractéristiques communes à la classe:
\begin{enumerate}
    \item des attributs\index{attribut}, des variables propres à aux instances;
    \item des méthodes, des fonctions propres aux instances et qui agissent par exemple sur leurs attributs.
\end{enumerate}


\subsubsection{Définition d'une classe}
Pour définir une classe, la syntaxe est la suivante:
\begin{minted}{python3}
class MaClasse:
    pass
\end{minted}

Lorsqu'on lance l'exécution d'un module Python, l'interpréteur exécute le contenu du corps de toutes les classes qu'il rencontre pour la première fois. Ainsi si l'on exécute :

\begin{minted}{python3}
class Foo:
    print('bar')
\end{minted}

La console affichera \mintinline{python3}{bar}. Pour créer une instance de classe, il faut appeler la classe. Appeler une classe revient à appeler son \hyperref[constructeur]{constructeur}, composé d'un créateur d'objet et d'un initialiseur. En reprenant la classe \mintinline{python3}{Foo}:

\begin{minted}{pycon}
>>> Foo
<class '__main__.Foo'>
>>> Foo()
<__main__.Foo object at 0x7f193dd99630>
\end{minted}

\subsubsection{Initialiseur}
\index{initialiseur}\index{\_\_init\_\_()}

L'initialiseur est une méthode spéciale appelée \mintinline{python3}{__init__()}, il prend en argument l'instance sur laquelle il est appliqué et tous les paramètres nécessaires à l'initialisation de l'instance. Par convention le premier paramètre, l'instance, est nommé \mintinline{python3}{self}.

\begin{minted}{python3}
class MaClasse:

    CONSTANTE = "Constante"

    def __init__(self, att1, att2):
        """Initialiseur"""
        self.attribut_1 = att1
        self.attribut_2 = att2
        self.attribut_3 = type(self).CONSTANTE
\end{minted}

Ici, les deux premiers attributs sont personnalisables lors de la création des objets alors que le dernier est commun à tous. \mintinline{python3}{MaClasse.CONSTANTE} est une variable de classe\index{variable de classe}. Pour créer un objet on écrit simplement:

\begin{minted}{python3}
objet = MaClasse(att1, att2)
\end{minted}

Les informations concernant les attributs d'une instance sont stockées dans le dictionnaire de l'instance, un attribut spécial nommé \mintinline{python3}{__dict__}.

\begin{minted}{pycon}
>>> instance = MaClasse("attr1", "attr2")
{'attribut_1': 'attr1', 'attribut_2': 'attr2', 'attribut_3': 'Constante'}
\end{minted}

\subsubsection{Méthodes}
\index{méthode}
Les méthodes se définissent comme des fonctions, elles agissent en général sur les instances de la classe. Elles doivent toujours prendre \mintinline{python3}{self} en argument :
\begin{minted}{python3}
class MaClasse:

    def __init__(self):
         pass

    def methode(self, arg1, arg2):
         pass
\end{minted}

Ensuite on les appelle de la manière suivante:
\begin{minted}{python3}
objet.methode(arg1, arg2)
\end{minted}


\subsection{Héritage}
\index{héritage}

\subsubsection{Principe}

L'héritage est un moyen de créer des classes dérivées (classes filles) d'une classe de base (classe mère). Une classe fille hérite de toutes les méthodes et variables de sa classe mère. Pour créer une classe fille, on utilise la syntaxe suivante.

\begin{minted}{python3}
class Mere:
    pass

class Fille(Mere):
    pass
\end{minted}

Il est possible d'écraser une méthode héritée en la redéfinissant dans la classe fille. Si on veut accéder à une méthode héritée alors qu'on l'a redéfinie dans la classe fille, on utilise la fonction \mintinline{python3}{super()} qui permet d'appeler la méthode de la classe mère de la classe présente (sans l'argument self).

\paragraph{Exemple}
\begin{minted}{python3}
class Meuble:
    def __init__(self, couleur, materiau):
        self.couleur = couleur
        self.materiau = materiau

class Bibliotheque(Meuble):
    def __init__(self, couleur, materiau, n):
        super().__init__(couleur, materiau)
        self.nb_livres = n
\end{minted}
On peut utiliser deux fonctions pour vérifier l'héritage: \mintinline{python3}{isinstance} renvoie \mintinline{python3}{True} si l'objet est une instance de la classe ou de ses classes filles ; \mintinline{python3}{issubclass} permet de voir si une classe est fille d'une autre.

\begin{minted}{pycon}
>>> bibli = Bibliotheque('blanc', 'vert', 150)
>>> bibli.__dict__
{'couleur': 'blanc', 'materiau': 'vert', 'nb_livres': 150}
>>> isinstance(bibli, Meuble)
True
>>> isinstance(bibli, Bibliotheque)
True
>>> issubclass(Bibliotheque, Meuble)
True
>>> issubclass(Meuble, Bibliotheque)
False
>>> isinstance(bibli, int)
False
>>> isinstance(bibli, object)
True
\end{minted}

\paragraph{Plus d'informations} \href{https://openclassrooms.com/courses/apprenez-a-programmer-en-python/l-heritage-9}{OpenClassrooms}, \href{https://docs.python.org/fr/3/tutorial/classes.html?highlight=héritage#inheritance}{Documentation Python 3}, \href{https://www.programiz.com/python-programming/inheritance}{Programiz}


\subsubsection{Ordre de résolution de méthode}

\subsubsection{Classe mère \mintinline{python3}{object}}
On a dit précédemment que le constructeur était composé d'un initialiseur et d'un créateur d'instance ; cependant l'exemple ne définissait pas de créateur d'instance : c'est parce qu'il est défini dans une classe \mintinline{python3}{object}. Toutes les classes en Python 3 héritent implicitement de cette classe. Elle définit de nombreuses méthodes, notamment les méthodes dites spéciales, que l'on peut surcharger pour les personnaliser.

\begin{minted}{pycon}
>>> class Foo:
...    pass
>>> issubclass(Foo, object)
True
>>> object
<class 'object'>
>>> help(object)
Help on class object in module builtins:

class object
 |  The most base type

>>> class Foo:
...    pass
...
>>> issubclass(Foo, object)
True
\end{minted}

\subsection{Propriétés}
\label{sec:proprietes}\index{propriété}\index{accesseur}\index{mutateur}\index{destructeur}
 Les propriétés représentent en Python le principe d'encapsulation. Elles sont utiles si on souhaite contrôler l'accès à un attribut ou si on veut que le changement d'une valeur d'un attribut engendre des modifications sur d'autres attributs. Les propriétés sont un cas particulier des descripteurs.\bigskip

On crée les propriétés en utilisant des décorateurs. Elles contiennent un accesseur, un mutateur, un destructeur et une aide (docstring de l'accesseur).\bigskip

Les propriétés sont aussi un moyen de simuler des attributs privés: pour simuler un attribut privé, on précède son nom d'un souligné. Ainsi, on appelle cet attribut sans le souligné dans le code grâce aux propriétés. Par convention, on n'agit pas sur les attributs qui commencent par un souligné en Python.

\paragraph{Exemple}~
\begin{minted}{python3}
class Temperature:
    def __init__(self, celsius):
        self._celsius = celsius
        self._fahrenheit = self._celsius * 1.8 + 32

    @property
    def celsius(self):
        """Propriété 'celsius'."""
        return self._celsius

    @celsius.setter
    def celsius(self, valeur):
        self._celsius = valeur
        self._fahrenheit = valeur * 1.8 + 32

    @celsius.deleter
    def celsius(self):
        del self._celsius
    
    @property
    def fahrenheit(self):
        """Propriété 'fahrenheit'."""
        return self._fahrenheit
    
    @fahrenheit.setter
    def fahrenheit(self, valeur):
        self._fahrenheit = valeur
        self._celsius = (valeur - 32) / 1.8

    @fahrenheit.deleter
    def fahrenheit(self):
        del self._fahrenheit
\end{minted}

On utilise la propriété de la manière suivante:
\begin{minted}{pycon}
>>> temp = Temperature(20)
>>> temp.fahrenheit
68.0
>>> temp.fahrenheit = 69
>>> temp.celsius
20.555555555555554
>>> del temp.fahrenheit
>>> temp.fahrenheit
Traceback (most recent call last):
...
AttributeError: 'Temperature' object has no attribute '_fahrenheit'
\end{minted}

\paragraph{Plus d'informations}\href{https://docs.python.org/fr/3/library/functions.html?highlight=property#property}{Documentation Python 3}, \href{https://stackoverflow.com/questions/15750522/class-properties-and-setattr/15751159#15751159}{Priorités entre propriété et méthodes spéciales}

\subsection{Méthodes statiques et méthodes de classes}
\subsubsection{Méthode statique}
\index{méthode statique}\label{sec:staticmethod}

Les méthodes que l'on a vues jusqu'à maintenant agissent sur les instances des classes : elles prennent toujours en premier argument le mot clé \mintinline{python3}{self} qui correspond à l'instance elle même. Lorsque l'on appelle une telle méthode sur une instance comme ceci:
\mint{python}|instance.methode(*args, **kwargs)|
Python exécute en fait: \mint{python}|type(instance).methode(instance, *args, **kwargs)|
Évaluer \mintinline{python3}{type()} sur un objet renvoie sa classe.\bigskip

En fait, ces deux objets sont différents. \mintinline{python3}{Classe.methode} est une simple fonction, alors que \mintinline{python3}{instance.methode} est une méthode partiellement évaluée sur l'instance (méthode liée, en anglais \og bound method \fg{}\index{bound method}), c'est-à-dire que l'instance est mise en premier argument.\bigskip

Parfois, on écrit des méthodes qui n'ont pas d'incidence sur les instances de la classe. Imaginons une classe \mintinline{python3}{Maths} qui regrouperait des opérations basiques:

\begin{minted}{python3}
class Maths:
    def addition(x, y):
        return x + y

    def multiplication(x, y):
        return x * y

    def division(x, y):
        return x / y
\end{minted}

\begin{minted}{pycon}
>>> Math.addition(1, 3)
4
\end{minted}

Jusqu'ici tout va bien. Mais imaginons que la classe se développe et que l'on instancie des objets \mintinline{text}{Maths}. On aura un soucis :

\begin{minted}{pycon}
>>> Maths().addition(1, 3)
Traceback (most recent call last):
  File "<stdin>", line 1, in <module>
TypeError: addition() takes 2 positional arguments but 3 were given
\end{minted}

Le problème étant que Python a en fait exécuté:

\mint{python3}|Maths.addition(Maths(), x, y)|

Pour remédier à cela, il existe le décorateur \mintinline{python3}{@staticmethod}. On doit écrire alors:

\begin{minted}{python3}
class Maths:
    @staticmethod
    def addition(x, y):
        return x + y

    @staticmethod
    def multiplication(x, y):
        return x * y

    @staticmethod
    def division(x, y):
        return x / y
\end{minted}

On peut ainsi écrire sans crainte:

\begin{minted}{pycon}
>>> Maths().addition(1, 3)
4
\end{minted}

\subsubsection{Méthode de classe}
\index{méthode de classe}\label{sec:classmethod}

Parfois, on veut pouvoir agir sur la classe et non sur l'instance. Dans ce cas, la méthode de classe prend en premier paramètre \mintinline{python3}{cls} (la classe) au lieu de \mintinline{python3}{self} (l'instance).

\begin{minted}{pycon}
>>> class Foo:
...     CONSTANT = "Bar"
...     def print_constant(cls):
...         print(cls.CONSTANT)
...
>>> Foo().print_constant()
Bar
\end{minted}

Deux problèmes surviennent. Tout d'abord, même si le premier paramètre s'appelle \mintinline{python3}{cls}, c'est encore l'instance qui est mise en paramètre.

\begin{minted}{pycon}
>>> foo = Foo()
>>> foo.CONSTANT = "Baz"
>>> foo.print_constant()
Baz # On veut Bar !
\end{minted}

Deuxième problème: on ne peut pas appeler la méthode de classe sur la classe (même problème que pour les méthodes statiques):

\begin{minted}{pycon}
>>> Foo.print_constant()
Traceback (most recent call last):
  File "<stdin>", line 1, in <module>
TypeError: print_constant() missing 1 required positional argument: 'cls'
\end{minted}

Pour remédier à cela, on utilise le décorateur \mintinline{python3}{@classmethod}.

\begin{minted}{pycon}
>>> class Foo:
...     CONSTANT = "Bar"
...     @classmethod
...     def print_constant(cls):
...         print(cls.CONSTANT)
...
>>> foo = Foo()
>>> foo.CONSTANT = "Baz"
>>> foo.print_constant()
Bar # Youpi !
>>> Foo.print_constant()
Bar # Youyoupi !
\end{minted}

\subsubsection{Cas de l'héritage}
En résumé:
\begin{enumerate}
        \item Les méthodes statiques sont des fonctions reliées à des classes, mais qui n'agissent pas sur celles-ci.
        \item Les méthodes de classe sont des fonctions qui prennent la classe en paramètre.
\end{enumerate}

Une classe qui hérite d'une classe mère hérite de toutes les méthodes de celle-ci. Les méthodes statiques restent donc inchangées, tandis que les méthodes de classe s'adaptent à la nouvelle classe, car elles la prennent en premier argument.

\paragraph{Exemple} Un exemple d'utilisation de méthodes statiques et de classe sont la création de constructeurs alternatifs. On s'aperçoit de la différence des deux notions.
\begin{minted}{python3}
class Personne:
    def __init__(self, nom, age):
        self.nom = nom
        self.age = age

    @staticmethod
    def par_date_de_naissance(nom, date):
        return Personne(nom, 2018-date)

    @classmethod
    def par_date_de_naissance2(cls, nom, date):
        return cls(nom, 2018-date)

class Homme(Personne):
    sexe = 'homme'
\end{minted}
\begin{minted}{pycon}
>>> homme1 = Homme.par_date_de_naissance('Jean', 1997)
>>> homme2 = Homme.par_date_de_naissance2('Jean', 1997)
>>> type(homme1)
<class '__main__.Personne'>
>>> type(homme2)
<class '__main__.Homme'>
\end{minted}

Pour avoir \mintinline{python3}{homme1} de type \mintinline{python3}{Homme}, il faut redéfinir la méthode statique dans la classe fille.

\paragraph{Plus d'informations} \href{https://www.programiz.com/python-programming/methods/built-in/staticmethod}{Méthode statique sur Programiz}, \href{https://www.programiz.com/python-programming/methods/built-in/classmethod}{Méthode de classe sur Programiz}, \href{https://stackoverflow.com/questions/136097/what-is-the-difference-between-staticmethod-and-classmethod-in-python/1669524#1669524}{StackOverflow}
