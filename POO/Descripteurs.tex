\section{Descripteurs}
Les descripteurs sont une généralisation des propriétés (plus précisément, les propriétés sont des descripteurs), ce sont
des objets qui doivent implémenter au moins une des méthodes spéciales \mintinline{python3}{__get__()}, \mintinline{python3}{__set__()}, \mintinline{python3}{__del__()}. Elles sont appelées lorsqu'on accède à l'attribut ou lorsqu'on essaye d'en
changer la valeur. Les descripteurs sont utiles pour créer des \og propriétés générales \fg{} qui n'ont pas besoin d'en savoir
beaucoup sur les classes propriétaires des attributs et qui peuvent être utilisées par différents types d'objets.

\paragraph{Exemple} Un descripteur d'entier borné
\begin{minted}{python3}
class BoundedIntDescriptor:
    """Entier borné.
    
    Vérifie que l'attribut est compris entre mini et maxi.
    """
    
    def __init__(self, mini, maxi, doc=None):
        self.min = mini
        self.max = maxi
        self.__doc__ = doc

    def __get__(self, inst, owner):
        print("--> __get__ du descripteur appelé.")
        return getattr(inst, '_' + self.name)

    def __set__(self, inst, value):
        print("--> __set__ du descripteur appelé.")
        if not self.min <= value <= self.max:
            raise ValueError("{} doit être compris entre {} et {} ({} donné).".format(
                self.name, self.min, self.max, value
            ))
        setattr(inst, '_' + self.name, value)
    
    def __set_name__(self, owner, name):
        self.name = name


class Time:
    def __init__(self, h, m, s):
        self.h = h
        self.m = m
        self.s = s

    h = BoundedIntDescriptor(0, 23, "Entier compris entre 0 et 23 représentant les heures.")
    m = BoundedIntDescriptor(0, 59, "Entier compris entre 0 et 59 représentant les minutes.")
    s = BoundedIntDescriptor(0, 59, "Entier compris entre 0 et 59 représentant les secondes.")

    def __repr__(self):
        return "{} h {} min {} s".format(self._h, self._m, self._s)
\end{minted}

Le recours aux fonctions \mintinline{python3}{getattr()} et \mintinline{python3}{setattr()} permet d'accéder aux attributs
des instances concernées. Si on avait stocké l'attribut dans l'instance du descripteur par exemple avec un \mintinline{python3}
{self.attr} au lieu de \mintinline{python3}{inst._attr}, changer par exemple la valeur de \mintinline{text}{h} pour \mintinline
{text}{t1} (cf. l'exemple précédent) aurait affecté \mintinline{text}{t2}! En effet, le descripteur est instancié \emph{au
moment où la classe est définie et non à l'initialisation des instances de celle-ci}. En résumé, tous les attributs \mintinline
{text}{h} pointent vers la même instance de \mintinline{text}{BoundedIntDescriptor}, de même pour \mintinline{text}{m}
et \mintinline{text}{s}.\bigskip

Jouons avec notre descripteur:

\begin{minted}{pycon}
>>> t1 = Time(1, 2, 3)
--> __set__ du descripteur appelé.
--> __set__ du descripteur appelé.
--> __set__ du descripteur appelé.
>>> t1
1 h 2 min 3 s
>>> t1.h = 6
--> __set__ du descripteur appelé.
>>> t1
6 h 2 min 3 s
>>> t2 = Time(10, 11, 12)
--> __set__ du descripteur appelé.
--> __set__ du descripteur appelé.
--> __set__ du descripteur appelé.
>>> t2
10 h 11 min 12 s
>>> t1
6 h 2 min 3 s
>>> help(t1)
--> __get__ du descripteur appelé.
--> __get__ du descripteur appelé.
--> __get__ du descripteur appelé.

Help on Time in module __main__ object:

class Time(builtins.object)
 |  Time(h, m, s)
 |  
 |  Methods defined here:
 |  
 |  __init__(self, h, m, s)
 |      Initialize self.  See help(type(self)) for accurate signature.
 |  
 |  __repr__(self)
 |      Return repr(self).
 |  
 |  ----------------------------------------------------------------------
 |  Data descriptors defined here:
 |  
 |  __dict__
 |      dictionary for instance variables (if defined)
 |  
 |  __weakref__
 |      list of weak references to the object (if defined)
 |  
 |  h
 |      Entier compris entre 0 et 23 représentant les heures.
 |  
 |  m
 |      Entier compris entre 0 et 59 représentant les minutes.
 |  
 |  s
 |      Entier compris entre 0 et 59 représentant les secondes.

>>> t1.m = -1
--> __set__ du descripteur appelé.
Traceback (most recent call last):
    File "<stdin>", line 1, in <module>
    File "<string>", line 19, in __set__
ValueError: m doit être compris entre 0 et 59 (-1 donné).
\end{minted}

\paragraph{Plus d'informations} \href{https://docs.python.org/3/howto/descriptor.html}{Documentation Python 3}
