\section{Descripteurs}\label{sec:descripteur}
Les descripteurs sont une généralisation des propriétés (plus précisément, les propriétés sont des descripteurs), ce sont
des objets qui doivent implémenter au moins une des méthodes spéciales suivantes:

\begin{description}
    \item[\mintinline{python3}{Descripteur.__get__(self, inst, owner)}]~
    
    Appelée lorsqu'on essaie d'accéder à l'attribut:
    \begin{minted}{python3}
inst.descripteur
    \end{minted}
    \mintinline{text}{inst} est l'instance sur laquelle on essaie d'accéder à l'attribut et \mintinline{text}{owner}
    est le type de \mintinline{text}{inst}, la classe propriétaire du descripteur. Dans le cas particulier où l'on
    accède à un attribut par la classe:
    \begin{minted}{python3}
Class.descripteur
    \end{minted}
    \mintinline{text}{inst} vaut \mintinline{python3}{None} et \mintinline{text}{owner} est la classe en question.\medskip

    Cette méthode doit retourner l'attribut en question ou le calcul de sa valeur, ou bien lever une exception \mintinline{python3}{AttributeError}.

    \item[\mintinline{python3}{Descripteur.__set__(self, inst, value)}]~
    
    Appelée lorsqu'on essaie de modifier la valeur d'un attribut d'\mintinline{text}{inst} de la classe propriétaire à
    \mintinline{text}{value}:
    \begin{minted}{python3}
inst.descripteur = value
    \end{minted}
    
    \item[\mintinline{python3}{Descripteur.__delete__(self, inst)}]~
    
    Appelée lorsqu'on essaie de supprimer l'attribut de \mintinline{text}{inst} de la classe propriétaire:
    \begin{minted}{python3}
del inst.descripteur
    \end{minted}
\end{description}

Les descripteurs sont utiles pour créer des \og propriétés générales \fg{} qui n'ont pas besoin d'en savoir
beaucoup sur les classes propriétaires des attributs et qui peuvent être utilisées par différents types d'objets.

\paragraph{Exemple} Un descripteur d'entier borné
\begin{minted}{python3}
class BoundedIntDescriptor:
    """Entier borné.
    
    Vérifie que l'attribut est compris entre mini et maxi.
    """
    
    def __init__(self, mini, maxi, doc=None):
        self.min = mini
        self.max = maxi
        self.__doc__ = doc

    def __get__(self, inst, owner):
        print("--> __get__ du descripteur appelé.")
        return getattr(inst, '_' + self.name)

    def __set__(self, inst, value):
        print("--> __set__ du descripteur appelé.")
        if not self.min <= value <= self.max:
            raise ValueError("{} doit être compris entre {} et {} ({} donné).".format(
                self.name, self.min, self.max, value
            ))
        setattr(inst, '_' + self.name, value)
    
    def __set_name__(self, owner, name):
        self.name = name


class Time:
    def __init__(self, h, m, s):
        self.h = h
        self.m = m
        self.s = s

    h = BoundedIntDescriptor(0, 23, "Entier compris entre 0 et 23 représentant les heures.")
    m = BoundedIntDescriptor(0, 59, "Entier compris entre 0 et 59 représentant les minutes.")
    s = BoundedIntDescriptor(0, 59, "Entier compris entre 0 et 59 représentant les secondes.")

    def __repr__(self):
        return "{} h {} min {} s".format(self._h, self._m, self._s)
\end{minted}

Le recours aux fonctions \mintinline{python3}{getattr()} et \mintinline{python3}{setattr()} permet d'accéder aux attributs
des instances concernées. Si on avait stocké l'attribut dans l'instance du descripteur par exemple avec un \mintinline{python3}{self.attr} au lieu de \mintinline{python3}{inst._attr}, changer par exemple la valeur de \mintinline{text}{h} pour \mintinline{text}{t1} (cf. l'exemple précédent) aurait affecté \mintinline{text}{t2}! En effet, le descripteur est instancié \emph{au moment où la classe est définie et non à l'initialisation des instances de celle-ci}. En résumé, tous les attributs \mintinline{text}{h} pointent vers la même instance de \mintinline{text}{BoundedIntDescriptor}, de même pour \mintinline{text}{m} et \mintinline{text}{s}.\medskip

Pour récupérer le nom de l'attribut afin d'afficher une erreur explicite, j'ai utilisé la méthode \mintinline{python3}{__set_name__()}.

\begin{description}
    \item[\mintinline{python3}{Descripteur.__set_name__(self, owner, name)}]~
    
    Appelée lors de la création de la classe, c'est-à-dire quand le descripteur est instancié. Le nom de l'instance de \mintinline{text}{Descripteur} est assigné à \mintinline{text}{name}.
\end{description}

Jouons avec notre descripteur:

\begin{minted}{pycon}
>>> t1 = Time(1, 2, 3)
--> __set__ du descripteur appelé.
--> __set__ du descripteur appelé.
--> __set__ du descripteur appelé.
>>> t1
1 h 2 min 3 s
>>> t1.h
--> __get__ du descripteur appelé.
1
>>> t1.m
--> __get__ du descripteur appelé.
2
>>> t1.s
--> __get__ du descripteur appelé.
3
>>> t1.h = 6
--> __set__ du descripteur appelé.
>>> t1
6 h 2 min 3 s
>>> t2 = Time(10, 11, 12)
--> __set__ du descripteur appelé.
--> __set__ du descripteur appelé.
--> __set__ du descripteur appelé.
>>> t2
10 h 11 min 12 s
>>> t1
6 h 2 min 3 s
>>> help(t1)
--> __get__ du descripteur appelé.
--> __get__ du descripteur appelé.
--> __get__ du descripteur appelé.

Help on Time in module __main__ object:

class Time(builtins.object)
 |  Time(h, m, s)
 |  
 |  Methods defined here:
 |  
 |  __init__(self, h, m, s)
 |      Initialize self.  See help(type(self)) for accurate signature.
 |  
 |  __repr__(self)
 |      Return repr(self).
 |  
 |  ----------------------------------------------------------------------
 |  Data descriptors defined here:
 |  
 |  __dict__
 |      dictionary for instance variables (if defined)
 |  
 |  __weakref__
 |      list of weak references to the object (if defined)
 |  
 |  h
 |      Entier compris entre 0 et 23 représentant les heures.
 |  
 |  m
 |      Entier compris entre 0 et 59 représentant les minutes.
 |  
 |  s
 |      Entier compris entre 0 et 59 représentant les secondes.

>>> t1.m = -1
--> __set__ du descripteur appelé.
Traceback (most recent call last):
    File "<stdin>", line 1, in <module>
    File "<string>", line 19, in __set__
ValueError: m doit être compris entre 0 et 59 (-1 donné).
\end{minted}

À ce stade, on peut se demander quelle solution choisir pour contrôler les accès aux attributs ou faire des attributs dynamiques: propriétés, descripteurs, méthodes spéciales ? La réponse est: si on peut faire ce que l'on veut facilement, ne pas choisir une solution compliquée inutilement! Une réponse sur StackOverflow (voir \og Plus d'informations \fg{}) propose ceci:
\begin{itemize}
    \item La première solution la plus simple: utiliser le mécanisme par défaut qui utilise l'attribut spécial \mintinline{python3}{__dict__} de l'instance.
    \item Si ce mécanisme est insuffisant, par exemple si on veut déclencher des calculs ou ajouter de la logique, utiliser les propriétés.
    \item Si ce mécanisme est insuffisant, écrire des descripteurs adaptés; on vient de voir qu'ils sont utiles pour des propriétés génériques par exemple.
    \item Si ce mécanisme est inadapté, utiliser \mintinline{python3}{__getattr__()}; cette méthode est utile pour simuler
    la présence d'attributs. Cette méthode n'agit pas sur les attributs existants.
    \item En dernier recours, utiliser \mintinline{python3}{__getattribute__()}; la différence avec la solution précédente est
    que cette méthode est la première appelée lorsque l'on accède à un attribut. Ainsi, on a la main sur absolument tous les
    attributs, ce qui peut engendrer des comportements indésirables... À utiliser avec précaution, donc.
\end{itemize}

\paragraph{Plus d'informations} \begin{itemize}
    \item \href{https://docs.python.org/fr/3/howto/descriptor.html}{Guide sur les descripteurs} (documentation officielle).
    \item \href{https://stackoverflow.com/a/22617259/9214306}{Quand utiliser les propriétés et les méthodes spéciales}
\end{itemize}