\section{Itérateurs}\label{iterateur}
\index{itérateur}

Les itérateurs sont des objets incontournables en Python, ils sont notamment utilisés lorsque l'on fait une boucle \mintinline{python3}{for}. Les objets itérateurs peuvent être créés par des objets itérables. Des itérables connus sont les listes, les dictionnaires, les tuples, les \mintinline{python3}{range()}.\bigskip

L'intérêt principal des itérateurs est leur faible consommation mémoire: contrairement à un objet conteneur qui prend autant d'espace que d'objets qu'il contient, un itérateur calcule chaque élément lorsqu'il est appelé.

\paragraph{Exemple} Ce qu'il se passe lorsque l'on fait une boucle \mintinline{python3}{for}
\begin{minted}{python3}
>>> L = [0, 1, 2, 3, 4] # les listes sont des objets itérables
>>> for element in L # appelle l'itérateur de l'itérable L
...    print(element) # à chaque ligne, appelle l'élément suivant de l'itérateur
0
1
2
3
4
\end{minted}

Les itérateurs sont implémentés sous forme de classes et doivent respecter le protocole d'itérateur: deux méthodes spéciales doivent être implémentées.
\begin{description}
    \item[\mintinline{python3}{iterateur.__iter__()}]~

    Cette méthode doit retourner l'itérateur lui-même, on peut auparavant y effectuer quelques opérations d’initialisation. Appel:
    \begin{minted}{python3}
iter(iterateur)
    \end{minted}

    \item[\mintinline{python3}{iterateur.__next__()}]~

    Cette méthode retourne l'élément suivant dans la séquence de l'itérateur. Une fois que le dernier élément a été appelé, lève une exception \mintinline{python3}{StopIteration}. Appel:
    \begin{minted}{python3}
next(iterateur)
    \end{minted}
\end{description}

Les itérables doivent quant à eux implémenter la méthode \mintinline{python3}{__iter__()} qui appelle l'itérateur associé. On l'appelle en faisant \mintinline{python3}{iter(objet_iterable)}.

\paragraph{Exemple} Un incrémenteur
\begin{minted}{python3}
class Incrementor:
    def __init__(self, max):
        self.max = max

    def __iter__(self):
        self. n = 0
        return self

    def __next__(self):
        if self.n <= self.max:
            result = self.n
            self.n += 1
            return result
        else:
            raise StopIteration
\end{minted}
On peut maintenant utiliser l'itérateur.
\begin{minted}{python3}
>>> inc = Incrementor(2)
>>> iterator = iter(inc)
>>> next(inc)
0
>>> next(inc)
1
>>> next(inc)
2
>>> next(inc)
Traceback (most recent call last):
  File "<stdin>", line XX, in <module>
    print(next(iterator))
  File "<stdin>", line XX, in __next__
    raise StopIteration
StopIteration
\end{minted}
On peut aussi utiliser une boucle \mintinline{python3}{for} pour itérer notre itérateur.
\begin{minted}{python3}
>>> for i in Incrementor(5):
...    print(i)
0
1
2
3
4
5
\end{minted}