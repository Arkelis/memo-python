\section{Fonctions et objets {\normalfont\bfseries\itshape callable}}\label{sec:callable}
Un objet est dit \textit{callable} (que l'on peut traduire par \og appelable \fg{}) si lui appliquer des parenthèses déclenche quelque chose.

\begin{minted}{python3}
obj() # on appelle l'objet obj
obj(arg) # on lui passe un argument positionnel
obj(arg1, param=arg2) # on lui passe aussi un argument nommé
result = obj() 
# En général appeler un objet retourne quelque chose. En fait, un callable renvoie toujours
# au moins None.
\end{minted}

\paragraph*{Plus d'informations}
\begin{itemize}
    \item \href{https://docs.python.org/3/reference/datamodel.html\#index-32}{les fonctions} (documentation Python 3);
    \item \href{http://sametmax.com/quest-ce-que-lunpacking-en-python-et-a-quoi-ca-sert/}{déballage} et \href{http://sametmax.com/quest-ce-quun-callable-en-python/}{callables} (Sam \& Max);
    \item \href{https://stackoverflow.com/a/32856533/9214306}{fonctionnement interne des callables} (StackOverflow).
\end{itemize}

\subsection{Créer un objet {\normalfont\bfseries\itshape callable}}

Pour qu'une classe puisse instancier des objets \emph{callable}, il suffit qu'elle définisse une méthode
\mintinline{python3}{__call__()}.

\begin{description}
    \item[\mintinline{python3}{Callable.__call__(self[, *args, **kwargs])}]~
 
    Cette méthode est appelée lorsqu'on appelle un objet. Tout objet possédant cette méthode est \emph{callable}.
\end{description}

\begin{minted}{pycon}
>>> class HelloCallable:
...     def __call__(self):
...         return "Hello"
...
>>> HelloCallable()() # () pour instancier puis () pour appeler
Hello
\end{minted}


Les classes sont \emph{callable}: on les appelle pour instancier.
\begin{minted}{pycon}
>>> class Bar: pass
...
>>> hasattr(Bar, "__call__")
True
\end{minted}

\subsection{Les objets fonctions}

Les fonctions et les méthodes sont des objets \textit{callable}, elles sont créées avec le mot-clé \mintinline{python3}{def}. On peut utiliser des annotations pour indiquer les types des paramètres et le type de la valeur renvoyée (les annotations peuvent être plus généralement n'importe quelle expression Python).

\begin{minted}{pycon}
>>> def func(arg1: int = 1) -> int:
...     """Docstring."""
...     return arg1 # l'interpréteur s'arrête au premier return rencontré
\end{minted}

Les fonctions sont des objets; à ce titre, on peut parfaitement leur définir des attributs de fonction. Un exemple farfelu
pour montrer que ça existe:

\begin{minted}{pycon}
>>> def incrementor():
...     incrementor.n += 1
...     return incrementor.n
...
>>> incrementor.n = -1
>>> incrementor()
0
>>> incrementor()
1
\end{minted}

Ces objets sont des instances de \mintinline{python3}{function}, ils possèdent quelques attributs spéciaux:

\begin{minted}{pycon}
>>> type(func)
<class 'function'>
>>> func.__name__
'func'
>>> func.__module__
'main'
>>> func.__annotations__
{'arg1': <class 'int'>, 'return': <class 'int'>}
>>> func.__defaults__
(1,)
>>> func.__doc__
'Docstring.'
\end{minted}

Les méthodes ont en plus un attribut spécial \mintinline{python3}{__self__} et un attribut spécial \mintinline{python3}{__func__}. On peut aussi créer une fonction anonyme avec \mintinline{python3}{lambda}:

\begin{minted}{pycon}
>>> func = lambda x: x # fonction identité (syntaxe -> lambda <args>: <expression à renvoyer>)
>>> type(func)
<class 'function'>
>>> func.__name__
'<lambda>'
\end{minted}

Ce mécanisme est utile si l'on veut créer une fonction à la volée, par exemple quand on utilise \mintinline{python3}{map()}.

\subsection{Gérer les paramètres d'une fonction}

On peut assigner à certains paramètres une valeur par défaut:

\begin{minted}{pycon}
>>> def function(arg1, arg2=5):
...     return arg1 + arg2
... 
>>> function(1)
6
\end{minted}

Les paramètres ayant une valeur par défaut doivent être placé \emph{après} ceux qui n'en ont pas.

\paragraph*{Remarque} Les valeurs par défaut sont assignées lors de la définition de la fonction. Si l'objet associé est
muable, la valeur par défaut restera toujours celle d'origine même si la fonction est appelée alors que l'objet associé
a mué.\medskip

On peut passer deux types d'arguments à une fonction:
\begin{enumerate}
    \item les arguments dits \emph{positionnels}: ils sont interprétés par la fonction grâce à leur position dans
          la liste des arguments passés à la fonction;
    \item les arguments dits \emph{nommés}: on spécifie le nom de leur paramètre quand on les passe à la fonction.
          On doit avoir passé tous les arguments positionnels avant de passer des arguments nommés.
\end{enumerate}

On peut utiliser les opérateurs d'\emph{unpacking} (\og déballage \fg{}) :
\begin{itemize}
    \item Lorsque l'on écrit les paramètres pour capter tous les paramètres possibles.
    \item Pour renseigner les arguments.
\end{itemize}

Cette fonction prend tous les arguments qui lui sont passés:
\begin{minted}{pycon}
>>> def fonction(*args, **kwargs):
...     print(locals())
...
>>> fonction(1, 2, 3, 'A', 'B', 'C', kwarg1=4, kwarg2='D')
{'args': (1, 2, 3, 'A', 'B', 'C'), 'kwargs': {'kwarg1': 4, 'kwarg2': 'D'}}
\end{minted}

\mintinline{python3}{*args} va contenir tous les arguments positionnels dans un tuple et \mintinline{python3}{**kwargs}
tous les arguments nommés dans un dictionnaire. Cette syntaxe est utilisée quand on définit une fonction qui en enrobe une
autre pour lui transmettre tous les arguments (voir les \hyperref[sec:decorateur]{décorateurs}).\medskip

On peut aussi utiliser le déballage lors d'appels de fonctions:
\begin{minted}{pycon}
>>> def fonction(par1, par2, par3, par4):
...     print(locals())
...
>>> arg1, arg2, arg3, arg4 = 1, 2, 3, 4
>>> tuple_args = (arg1, arg2)
>>> dict_args = {'par3': arg3, 'par4': arg4}
>>> fonction(*tuple_args, **dict_args)
{'par1': 1, 'par2': 2, 'par3': 3, 'par4': 4}
\end{minted}

\paragraph*{Remarque} Le déballage, ce n'est pas que pour les paramètres de fonctions:
\begin{minted}{pycon}
>>> L = [1, 2, 3]
>>> a, b, c = L # on déballe L
>>> print(a, b, c)
1 2 3
>>> char1, char2 = "12" # ça fonctionne pour tout itérable
>>> print(char1, char2)
1 2
>>> a, b = (b, a)
>>> a, b = b, a # équivalent à la ligne précédente, l'assignation simultanée est en fait du déballage
>>> d = {'a': 1, 'b': 2, 'c': 3}
>>> for k, v in d.items():
...     print(k, ":", v)
... 
a : 1
b : 2
c : 3
\end{minted}
