\section{Fonctions et objets appelables}
La façon la plus basique de définir une fonction est d'utiliser le mot-clé \mintinline{python3}{def}. Lorsque l'on appelle, on spécifie des arguments aux paramètres de la fonction.

\begin{minted}{python3}
fonction(argument_de_paramètre_non_nommé, paramètre_nommé=argument)
\end{minted}

On peut utiliser les opérateurs d'\emph{unpacking} (\og déballage \fg{}) :
\begin{itemize}
    \item Lorsque l'on écrit les paramètres pour capter tous les paramètres possibles.
    \item Pour renseigner les arguments.
\end{itemize}

\paragraph{Exemple}
\begin{minted}{python3}
# Cette fonction accepte tous les arguments
def fonction(*args, **kwargs):
    pass

# Celle-ci en accepte 4. On va tester sur celle-ci l'unpacking.
def fonction(par1, par2, par3, par4):
    for (arg, value) in locals().items():
        print(arg, ':', value)

arg1, arg2, arg3, arg4 = 1, 2, 3, 4
tuple_args = (arg1, arg2)
dict_args = {'par3': arg3, 'par4': arg4}

# Testons l'unpacking
fonction(*tuple_args, **dict_args)

# Sortie
# par4 : 4
# par3 : 3
# par2 : 2
# par1 : 1
\end{minted}